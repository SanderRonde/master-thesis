\chapter{Conclusion}\label{chap:conclusion}

In this case study, a number of aspects of migrating a set of Angular components to Web Components are evaluated. These aspects include the question whether the migration process is possible at all, what a possible performance impact is, and how the resulting migrated components relate to other component libraries in the field. Chapter~\ref{chap:case-study} describes the various issues we faced during this project, eventually showing that it is possible to migrated a set of Angular components to Web Components. In Chapter~\ref{chap:results} we evaluate the end resulting Web Components, comparing them to both the original Angular components they were created from and various other UI libraries. We find that the newly created Web Components are only slightly slower in rendering and take only about twice as much time to load. The resulting components hold up very well compared to other UI libraries, initially being faster than quite a few of them but becoming relatively slower as the number of rendered components increases. This means that although the Web Components library was migrateded from Angular components, it can compete quite well with other UI libraries that were written from scratch. This confirms the technical feasibility of migrating Angular components to Web Components.

Further, when looking at the business side in Section~\ref{sec:results:time-spent} and Chapter~\ref{chap:discussion}, we find that the impact on the existing codebase and other developers is minimal. We also find the time spent on this migration to be definitely worth it depending on the time required to build the UI library from scratch combined with the time taken to maintain the UI library and adding new components. This indicates that this migration is a worthwhile investment, leading to freedom from maintaining two sets of components and the ability to easily add a new component to the Web Components library without issues. It should be noted that these results are not based on sufficient data to draw definitive conclusions. Instead, we indicate that these were our results and leave definitive proof for future work.

Further future work could be done on improving the performance of the created Web Components. Minimizing the performance impact of this migration process would aid in making these components just as viable as components written from scratch. Additionally, an interesting area to focus on is the list of issues faced during the migration process. Future work could revolve around performing another such case study and comparing the issues faced, extracting a list of issues that are always faced during this migration process. Such a list would allow for the creation of structural solutions to these problems, for example in the form of a freely downloadable JS package that aids in the migration process.

Finally, one shortcoming of this thesis is that we were only able to evaluate the effectiveness of migrating \textbf{Angular} components to Web Components. Not the migration of components from any JS framework to Web Components. We conjecture that this process should be just as feasible, with other frameworks likely taking significantly less time to migrated than Angular. As such, we believe further research into the migration of components from other JS frameworks to Web Components would be very beneficial, eventually leading to a situation where we can conclude that components from all JS frameworks can be migrated to Web Components, eventually making them re-usable across all JS frameworks. Furthermore, future work could be done on improving the performance of the created Web Components. Lastly,