\chapter{Conclusion}

In this case study the feasibility of the converting of a set of Angular components to Web Components was evaluated. Chapter~\ref{chap:case-study} describes the various issues we faced during this project, eventually showing that it is possible to convert a set of Angular components to Web Components. In Chapter~\ref{chap:results} we evaluate the end resulting Web Components, comparing them to both the original Angular components they were created from and various other UI libraries. We find that the newly created Web Components are only slightly slower in rendering and take only about twice as much time to load. The resulting components hold up very well compared to other UI libraries, being faster than quite a few of them. This means that regardless of the fact that the Web Components library were converted from Angular components, it is able to compete with other UI libraries that were written from scratch. This confirms the technical feasibility of the converting of Angular components to Web Components.

Further, when looking at the business side in Section~\ref{sec:results:time-spent} and Chapter~\ref{chap:discussion}, we find that the impact on the existing code base and other developers is minimal. We also find the timespent on this conversion to be definitely worth it depending on the time required to build the UI library from scratch combined with the time taken maintaining the UI library and adding new components. This shows that this conversion is a worthwhile investment, leading to freedom from having to maintain two sets of components and the ability to easily add a new component to the Web Components library without issues.

One shortcoming of this thesis is the fact that we were only able to evaluate the effectiveness of converting \textbf{Angular} components to Web Components. Not the conversion of components from any JS framework to Web Components. We theorize that this process should be just as feasible, with other frameworks likely taking significantly less time to convert than Angular. As such, we believe further research into the conversion of components from other JS frameworks to Web Components would be very benificial, eventually leading to a situation where we can conclude that components from all JS frameworks can be converted to Web Components, eventually making them re-usable across all JS frameworks.
\todo{Check loading time}
\todo{Take a look at the average render time of the other UI libraries, are we faster than the average?}