\chapter{Introduction}
As the number of JavaScript (JS) frameworks increases, the amount of fragmentation increases along with it~\footurl{https://insights.stackoverflow.com/survey/2020}. Packages such as UI libraries are almost always written for one specific JS framework and provide little to no compatibility with others. With the introduction of Web Components~\footurl{https://www.w3.org/TR/2013/WD-custom-elements-20130514/\#about}, the World Wide Web Consortium attempted to create a bridge between these frameworks. A single component format that would be able to be used in all JS frameworks. Web Components turned out to be a great solution, being supported in most major JS frameworks~\footurl{https://custom-elements-everywhere.com/}. Unfortunately however, most UI libraries and component libraries are still written in a framework other than Web Components. A method for converting these components to Web Components needs to be devised, such that components in a given framework can be made available to all other frameworks. We will be focusing on converting Angular components to Web Components.

With the introduction of Angular Elements~\footurl{https://angular.io/guide/elements}, a very easy and attractive method of converting Angular components to Web Components was created. The use of this JS library should allow for easy conversion from Angular components to Web Components, bridging the gap between Angular and all other frameworks.

In this paper, we will be evaluating the effectiveness of converting an existing set of Angular components to Web Components. Our approach to this is a case study where we convert an existing Angular component library to Web Components. To then bridge the gap between the created Web Components and other JS frameworks, we will be creating wrappers that ensure the created components can run in other JS frameworks, as well as generating documentation and individual component demo pages for developers. We will then be evaluating the feasibility of this conversion process. This assessment is done through the collection of various metrics. These will be collected on both the original Angular components, the Web Components version and the various wrappers, as well as a set of popular JS component libraries. We will then compare the created Web Components library to both its origin and other component libraries in the field, allowing us to assess the feasibility of this conversion process.

This paper is structured as follows: in Chapter~\ref{chap:background} an explanation as to the context of this case study is presented; in Chapter~\ref{chap:related-work} related work is discussed; in Chapter~\ref{chap:design} the study design is described; in Chapter~\ref{chap:experimental-setup} our approach to performing experiments is discussed; in Chapter~\ref{chap:case-study} our case study and the challenges we faced are documented; in Chapter~\ref{chap:results} the results of our metrics are laid out; in Chapter~\ref{chap:threats-to-validity} possible threats to validity are discussed; in Chapter~\ref{chap:discussion} we discuss our findings and what they can and can not tell us and Chapter~\ref{chap:conclusion} contains the conclusion.