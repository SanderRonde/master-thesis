\chapter{Discussion}\label{chap:discussion}

The results described in Chapter~\ref{chap:results} show that the creation of a UI library from an existing codebase is very well possible in an Angular application. Render times are only slightly higher, remaining competitive with various other UI libraries. One negative aspect seems to be that the render times increase pretty quickly with a higher number of components. Further, load times are not significantly higher in all cases except the Angular wrapper. This should result in a good user experience across the board, being slightly slower than the original components but providing access to them in the most popular JS frameworks. We can say that the answer to SRQ1 is that it is definitely technically feasible to convert Angular components to Web Components.

While the technical results of this project are important, we also evaluated the business side of this project through SRQ2. We find the time spent to be five months of FTE\@. We also took a look at the degree to which this project interferes with the original codebase and its developers' workflows through a questionnaire answered by the three front-end developers at 30MHz. It should be noted that of the three front-end developers, two indicate they only work on the front-end once a week or less. Additionally, the third front-end developer had recently joined the company and has not witnessed the complete development process described in this paper. As such, we are unable to draw definitive conclusions from the results of the questionnaire, and we instead treat the results as mere indications.

In questioning these three front-end developers at 30MHz, we found that, on average, they rated the impact of changes to the main codebase as a 2.6 on a scale from 0 (no impact at all) to 10 (significant impact). For a process that interlocks with the main codebase so heavily, this is a very low number, leading us to believe that the impact was slim.
Additionally, there are some new factors that developers have to keep in mind while developing new components. An example of this is the need for better documentation for components in order to ensure the automatically generated documentation is correct. Another example would be the need to add a new UI component to the array containing all components that are to be included in the UI library. On average, the developers rated the impact of these changes to be a 2, signaling that the everyday impact is not very large. Lastly, we asked developers how often the existence of this project blocked their workflow. They all indicated they had not been blocked once, meaning this project was executed entirely without blocking other developers' workflow. These results suggest that the answer to SRQ2 is that the business viability of the conversion of Angular components to a UI library is relatively high as well, leading to minimal impact on current developers and their workflow while requiring relatively little time. Especially in a situation where there are many UI components, the time spent on this project is significantly smaller than the time spent recreating them.

All in all, we can conclude that the answer to RQ1 is that the process of converting Angular components to a Web Component UI library is feasible. We hope this case study convinces businesses who are considering this process to take the steps we have taken over creating an entirely new UI library. In addition to being used in the manner we described, that is, the creation of a UI library for third parties, this process could also be applied to components that are internal to a business. With the ever-increasing amount of platforms with which users are able to interact (desktops, phones, tablets, televisions, smart fridges), the number of platforms for which businesses need to develop an application also increases. Since most of these platforms require different software stacks, Web Components could provide a basis for generating components for other platforms. For example, the main large web app can be built in Angular, with another small internal web app being built in React (using the React wrapper), another internal web app built in Vue, and the mobile apps built using React Native~\footurl{https://reactnative.dev/} or Apache Cordova~\footurl{https://cordova.apache.org/}.

What the results of this study do not tell us is the viability of converting components from any other JS framework to Web Components. In this case study, we specifically targeted Angular, which provides the simple Angular Elements tool~\footurl{https://angular.io/guide/elements}. Other JS frameworks might not have such tools available, which might make this process less straightforward. However, we believe that the process of converting components from any other popular JS framework to Web Components may very well be significantly easier than from Angular components. A large number of the issues we faced were Angular related, as described in Section~\ref{sec:case-study:ng-deep}. Those issues were also by far the hardest to solve. Most of these issues would not appear when using other JS frameworks.