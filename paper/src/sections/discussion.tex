\chapter{Discussion}\label{chap:discussion}

The results described in Chapter~\ref{chap:results} show that the creation of a UI library from an existing codebase is very well possible in an Angular application. Render times are only slightly higher, remaining competitive with various other UI libraries. One negative aspect seems to be that the render times increase quite quickly with a higher number of components. Further, load times are not significantly higher in all cases except the Angular wrapper. This all should result in a good user experience across the board, being slightly slower than the original components but providing access to them in most popular JS frameworks. We can say that the answer to SRQ1 is that it is definitely technically feasible to convert Angular components to Web Components.

While the technical results of this project are important, we also evaluated the business side of this project through SRQ2. We find the time spent to be five months of FTE\@. We also took a look at the degree in which this project interferes with the original codebase and its developers' workflows. In questioning the three front-end developers at 30MHz, we found that on average they rated the impact of changes to the main codebase as a 2.6 on a scale from 0 (no impact at all) to 10 (significant impact). For a process that interlocks with the main codebase so heavily, this is a very low number, leading us to believe that the impact was small. Additionally, there are some new factors that developers have to keep in mind while developing new components. An example of this is the need for better documentation for components in order to ensure the automatically generated documentation is correct. Another example would be the need to add a new UI component to the array containing all components that are to be included in the UI library. On average the developers rated the impact of these changes to be a 2, signaling that the every day impact is not very large. Lastly, we asked developers how often their workflow was blocked by the existence of this project. All of them indicate they have not been blocked once, meaning this project was executed completely in parallel and without blocking other developers' workflow. These results suggest that the answer to SRQ2 is that the business viability of the conversion of Angular components to a UI library is quite high as well, leading to minimal impact on current developers and their workflow, while requiring relatively little time. Especially in a situation where there are a large number of UI components, the time spent on this project is significantly smaller than the time spent recreating them.

All in all we can conclude that the answer to RQ1 is that the process of converting Angular components to a Web Component UI library is quite feasible. We hope this case study convinces businesses who are considering this process to take the steps we have taken over the creation of an entirely new UI library. In addition to being used in the manner we described, that is the creation of a UI library for 3rd parties, this process could also be applied to components that are internal to a business. With the ever increasing amount of platforms with which users are able to interact (desktops, phones, tablets, televisions, smart fridges), the number of platforms for which businesses need to develop an application also increases. Since most of these platforms require different software stacks, Web Components could provide a basis off of which to generate components for other platforms. For example the main large web app can be built in Angular, with another small internal web app being built in React (using the React wrapper), another internal web app built in Vue, and the mobile apps built using React Native~\footurl{https://reactnative.dev/} or Apache Cordova~\footurl{https://cordova.apache.org/}.

What the results of these study do not tell us is the viability of converting components from any other JS framework to Web Components. In this case study we specifically targeted Angular, which provides the simple Angular Elements tool~\footurl{https://angular.io/guide/elements}. Other JS frameworks might not have such tools available, which might make this process less straightforward. However, we believe that the process of converting components from any other popular JS framework to Web Components may very well be significantly easier than from Angular components. A large number of the issues we faced were Angular related, as described in Section~\ref{sec:case-study:ng-deep}. Those issues were also by far the hardest to solve. Most of these issues would not appear when using other JS frameworks.

As mentioned, this case study only allows us to draw conclusions of the viability of converting \textbf{Angular} components to Web Components. As such, we think a good starting point for further research is research into this same process for other JS frameworks. If this process is proven to be viable for other JS frameworks as well, businesses would be able to convert their components to Web Components regardless of their original JS framework, allowing for this process to be far more widely used.