\chapter{Threats to Validity}\label{chap:threats-to-validity}

In this chapter, we will be covering threats to the validity of this study. Firstly we will discuss the internal validity, after which we will discuss the external validity of the study.

\section{Internal Validity}
Possible internal threats to validity would be the measurement of our metrics being influenced by external factors. As described in Section~\ref{sec:experimental-setup:load-time} we explicitly remove the factor of network speed from our benchmarks. This leaves only the factor of available system resources as a possible variable. In order to eliminate this factor, we take several steps. We first ensure a clean testing environment by shutting down all unneeded background processes on the test machine. This should vastly reduce the amount of fluctuation in available system resources. Secondly, we ensure that only a single test is running at a time. This means every test has the entire computer to itself (in practice, likely a single core) and does not compete with other tests for system resources. Lastly, we apply all the steps described in Section~\ref{sec:experimental-setup:time-sensitive-metrics} which includes randomizing the order in which the tests are run and increasing the number of tests to \numMeasures{} measurements per test. This should ensure that any possible fluctuations are smoothed out and shared across all tests.

\section{External Validity}
While a large number of the problems faced in this case study are Angular-specific, a significant amount of them apply to Web Components in general, as shown in Section~\ref{sec:web-component-issues}. For this reason, the results in this study can be generalized to other JS frameworks as well. Further, while the specific UI library we created is dependent mainly on the 30MHz codebase and its specific architecture and contents, we make sure to compare the created CC UI libraries with the original 30MHz codebase itself, ensuring all results are relative to the original. This should ensure we answer the research question for a generalized case. If we were to compare the CC UI library solely to other UI libraries, the answer to the research question would only apply to this specific case.