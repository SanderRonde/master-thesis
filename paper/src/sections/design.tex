\chapter{Study Design}\label{chap:design}

\section{Research questions}
Our goal in this paper is to evaluate the effectiveness of migrating Angular components to Web Components. Based on this goal, we devise a single research question:
\\
\\
\textbf{RQ1: How technically viable is the process of migrating Angular components to Web Components?}
\\
In answering this research question, we assess whether the migration process is possible at all, what a possible performance impact could be, and how the resulting migrated components relate to other component libraries in the field.
\\

\section{Metric definitions}
In order to answer the above research questions, we need to define metrics. These allow us to compare the resulting Web Component library both to the UI library and to other UI libraries. This allows us to get a better understanding of the quality and performance of the created Web Component library relative to other UI libraries. We can divide the used metrics into two categories. The first is measuring the quality of the resulting CC UI library. In order to do measure the quality, we compare the components in the CC UI library to the Angular components they originate from, as well as to various UI libraries. We perform this comparison using various metrics divided into three groups. These groups are \emph{complexity}, \emph{size}, and \emph{performance}. We perform this comparison both at component granularity and at UI library granularity. A full list of these metrics, as well as a brief description, can be seen in Table~\ref{tab:design:metrics}. A detailed explanation of these metrics follows.


\begin{table}[htbp]
  \tiny
  \begin{tabularx}{\textwidth}{|l|l|l|l|X|}
    \toprule
    \textbf{ID} & \textbf{Group} \textbf{Metric} & \textbf{Granularity}   & \textbf{Description}                                                                                                                                                 \\ \midrule
    SC          & Complexity                     & Structural complexity  & Component                        & The number of import statements for a component. Collected for a source file and all of its dependencies for up to two iterations \\ \hline
    CC          & Complexity                     & Cyclomatic complexity  & Component                        & A quantitative measure of the number of linearly independent paths through a program's~\cite{1702388}                             \\ \hline
    LOC         & Size                           & Lines of code          & Component                        & The number of lines of code in a given component's source file                                                                    \\ \hline
    SI          & Size                           & Size                   & UI Library                       & The file size of the bundled up library                                                                                           \\ \hline
    MA          & Performance                    & Maintainability        & Component                        & A derivative based on complexity, lines of code and Halstead volume~\cite{halstead1977elements}                                   \\ \hline
    RT          & Performance                    & Render Time            & Component                        & The render time of a given component                                                                                              \\ \hline
    LT          & Performance                    & Load Time              & UI Library                       & Parsing and running time of the bundled up library in the browser (without download time)                                         \\ \hline
    NOC         & Performance                    & Number of Components   & UI Library                       & The number of components in a UI library                                                                                          \\ \hline
    FC          & Performance                    & First Paint            & UI Library (cow-components only) & First paint event of the browser                                                                                                  \\ \hline
    FCP         & Performance                    & First Contentful Paint & UI Library (cow-components only) & First paint event of the browser that includes content for the user  (text, images, etc.)                                         \\
  \end{tabularx}
  \caption{Metrics used in this study}
  \label{tab:design:metrics}
\end{table}

\subsection{Source code metrics}
The first set of metrics, namely structural complexity, cyclomatic complexity, lines of code, and maintainability, are metrics that are recommended by Martinez-Ortiz \etal{}~\cite{martinez-ortiz2016quality} as described in Section~\ref{sec:related-work:metrics}. We use these metrics to compare the quality of our Web Components to other Web Components. We follow almost all recommendations by the paper, including collecting the structural complexity up to a depth of two. Note that we do things slightly differently from the paper. We also keep track of the lines of code metric, which the authors do not. We do not use the lines of code metric to compare the quality of Web Components, but instead we use it to get a rough overview of the complexity of various UI libraries. Note that we are also not using all metrics recommended by the authors. The metrics we are not using are the metrics completeness (i.e.~how complete the information displayed to the user is) and consistency (i.e.~how long it takes for data to update across different replicas). We are not using completeness because it does not apply at the level at which the CC UI library operates. All of our components are 100\% complete, as well as the components of the UI libraries we compare the CC UI library to. As such, it does not make for a very interesting metric. This metric is very effective when more complex components such as entire pages are concerned, but that is not the case here. We also do not use the consistency metric. The reason for this is relatively simple, namely that we do not have any components with the ability to update across different replicas. The same goes for the UI libraries with which we compare the CC UI library.

\subsection{Size}
The size metric aims to measure the theoretical impact of loading the UI library over the network. We measure this at UI library level granularity since the contributions of individual components are very hard to measure. This should serve as a good indication of the relative network loading time of UI libraries without introducing the variable of network speed. In order to differentiate between a relatively large library and a library that has many components, we also keep track of the number of components metric.

\subsection{Load Time}
The load time metric aims to provide a measure of the real impact of a UI library on the page by measuring the real-world load time. The load time we measure is the load time of the main JS bundle. This contains the code needed to register the components to the page, as well as the code that performs the rendering. Note that we only measure the parsing time and running time of the JavaScript bundle. We explicitly exclude the download time from this metric since this is already captured in SI\@. A more in-depth definition of how this metric is captured is laid out in section~\ref{sec:experimental-setup:load-time}.

\subsection{First Paint \& First Contentful Paint}
The first paint metric, along with the first contentful paint metric, are only collected for the CC UI libraries. These metrics give us an indication of the real-world load time of a page containing the CC UI library and the original components. We use these metrics to evaluate how the paint time of the UI libraries has changed after its migration to Web Components and its later migration to JS framework wrappers. While these metrics do not serve as a perfect way to measure the perceived load time of a page, as discussed in Section~\ref{sec:related-work:load-time}, they should serve as an excellent comparison between the various distributions of the CC UI library. Since each of them contains the exact same content and is derived from the exact same source code, imperfections in these metrics are applied to all test subjects equally.

\subsection{Render Time}
Finally, the render time metric aims to capture the duration of the render cycle. We will define this render time as the time between setting the component's visibility to true and the browser completing the rendering process. If the render time of components in the CC UI library is significantly higher than components in other UI libraries or the Angular components they originate from, the performance impact of migrating Angular components to Web Components will be too significant. If it is slightly higher, the same, or lower, we can conclude that the performance impact is minimal.

In order to obtain an objective measurement of the rendering time that is independent of user perception, we chose to measure the render time of individual components. Since the components we use in our comparisons all load in a single stage (they are either not visible or visible), there is no loading state that could cause ambiguity.

\section{Metric targets}
In order to get a sense of the state of the CC UI library, we need to compare it to other UI libraries. To do so, we have gathered a list of various UI libraries targeting the most popular JS frameworks. Four of the most popular frameworks are ReactJS, Angular, Vue, and Svelte~\footurl{https://2020.stateofjs.com/en-US/technologies/front-end-frameworks/}. Through this comparison, we can compare the wrapper targeting a specific JS framework with UI libraries that also target that JS framework, allowing us to observe the influence of the framework itself on the various metrics. These UI libraries are gathered by searching for the terms ``Design Library'', ``UI Library'', ``javascript UI Library'', ``Svelte UI library'', ``React UI Library'', ``Vue UI Library'', ``Angular UI Library'', and ``Web Component UI Library'' on Google. We then add any UI library to the list that we came across, either by finding it as a direct result or it being mentioned in a blog post or article. A list of the UI libraries we found and the number of stars on their GitHub page can be found in Table~\ref{tab:design:ui-libraries}. While this is not a complete list of all UI libraries, we feel like it is an accurate representation of the most popular UI libraries since it contains all of the biggest UI libraries, as confirmed by the numerous blog posts listing them in order. In order to get a reasonably accurate representation of libraries using each JS framework, we select the three UI libraries with the most GitHub stars per JS framework. The list of included UI libraries can also be seen in Table~\ref{tab:design:ui-libraries}. In addition to comparing the CC UI library against other UI libraries, we also compare it against the Angular components from which they originate. We do this by applying our metrics to the 30MHz dashboard and the relevant components within it.

Since the components included in the selected UI libraries vary greatly, we can not make a proper comparison between individual components of the UI library. For example, a button component can not be immediately compared with a date picker component since date pickers tend to be more complex. In this scenario, higher rendering times can not be attributed to the UI library running it but to the component itself. In order to be able to compare every UI library, we have selected a set of basic components that are available in every UI library. These are the Button, Input (also known as TextField), and Switch (also known as Checkbox). Since every UI library we compare against contains all of these, we can compare the metrics for a single component across all UI libraries. We only apply the various metrics to these three components in each UI library. We also include a stripped-down version of the CC UI library in the set of UI libraries of which we gather metrics. This version only contains the three components mentioned above and allows for a fair comparison with other UI libraries. The reason for this is further explained in Section~\ref{sec:experimental-setup:size}.

\begin{table*}[t]
  \tiny{}
  \begin{tabularx}{\textwidth}{p{0.15\textwidth} |p{0.1\textwidth} | p{0.18\textwidth} | p{0.07\textwidth} | p{0.1\textwidth} |p{0.4\textwidth}  }
    \toprule
    \textbf{UI Library}     & \textbf{Github Stars}   & \textbf{JS Framework} & \textbf{Included} & \textbf{Version} & \textbf{Website}                                                                 \\ \midrule
    Svelte Material UI      & 1.6k                    & Svelte                & Yes               & 2.0.0            & \url{https://sveltematerialui.com/}                                              \\ \hline
    Smelte                  & 889                     & Svelte                & Yes               & 1.1.2            & \url{https://smeltejs.com/}                                                      \\ \hline
    Svelte-MUI              & 237                     & Svelte                & Yes               & 0.0.3-7          & \url{https://svelte-mui.ibbf.ru/}                                                \\ \hline
    Svelteit                & 51                      & Svelte                & No                & -                & \url{https://docs.svelteit.dev/}                                                 \\ \hline
    Material UI             & 67.1k                   & ReactJS               & Yes               & 5.0.0-alpha.28   & \url{https://material-ui.com/}                                                   \\ \hline
    React Bootstrap         & 19.2k                   & ReactJS               & Yes               & 1.5.2            & \url{https://react-bootstrap.github.io/}                                         \\ \hline
    React Semantic UI       & 12.2k                   & ReactJS               & Yes               & 2.0.3            & \url{https://react.semantic-ui.com/}                                             \\ \hline
    Evergreen               & 10.6k                   & ReactJS               & No                & -                & \url{https://evergreen.segment.com/}                                             \\ \hline
    Rebass                  & 7.2k                    & ReactJS               & No                & -                & \url{https://rebassjs.org/}                                                      \\ \hline
    Grommet                 & 7.1k                    & ReactJS               & No                & -                & \url{https://v2.grommet.io/}                                                     \\ \hline
    Baseweb                 & 6.2k                    & ReactJS               & No                & -                & \url{https://baseweb.design/}                                                    \\ \hline
    Ant Design              & 5.3k                    & ReactJS               & No                & -                & \url{https://ant.design/}                                                        \\ \hline
    Elemental UI            & 4.3k                    & ReactJS               & No                & -                & \url{http://elemental-ui.com/home }                                              \\ \hline
    Zendesk Garden          & 858                     & ReactJS               & No                & -                & \url{https://garden.zendesk.com/}                                                \\ \hline
    Shards React            & 649                     & ReactJS               & No                & -                & \url{https://designrevision.com/docs/shards-react/getting-started }              \\ \hline
    Angular Material        & 21.3k                   & Angular               & Yes               & 12.0.0-next.5    & \url{https://material.angular.io/}                                               \\ \hline
    NG-Bootstrap            & 7.7k                    & Angular               & Yes               & 9.1.0            & \url{https://ng-bootstrap.github.io/\#/home }                                    \\ \hline
    NGX-Bootstrap           & 5.3k                    & Angular               & Yes               & 7.0.0-rc.0       & \url{https://valor-softw2are.com/ngx-bootstrap/\#/ }                             \\ \hline
    NG-Lightning            & 886                     & Angular               & No                & -                & \url{https://ng-lightning.github.io/ng-lightning/\#/ }                           \\ \hline
    Alyle                   & 236                     & Angular               & No                & -                & \url{https://alyle.io/}                                                          \\ \hline
    Blox Material           & 143                     & Angular               & No                & -                & \url{https://material.src.zone/}                                                 \\ \hline
    Mosaic                  & 117                     & Angular               & No                & -                & \url{https://mosaic.ptsecurity.com/button/overview }                             \\ \hline
    Element                 & 49.8k                   & Vue                   & Yes               & 1.0.2-beta.40    & \url{https://element-plus.org/\#/en-US}                                          \\ \hline
    Vuetify                 & 30.2k                   & Vue                   & Yes               & 2.4.9            & \url{https://vuetifyjs.com/en/}                                                  \\ \hline
    Quasar                  & 18.3k                   & Vue                   & Yes               & 1.15.10          & \url{https://quasar.dev/ }                                                       \\ \hline
    Ant Design Vue          & 14.1k                   & Vue                   & No                & -                & \url{https://2x.antdv.com/docs/vue/introduce }                                   \\ \hline
    Bootstrap Vue           & 13.1k                   & Vue                   & No                & -                & \url{https://bootstrap-vue.org/ }                                                \\ \hline
    Vue-material            & 9.3k                    & Vue                   & No                & -                & \url{https://vuematerial.io/ }                                                   \\ \hline
    Buefy                   & 8.6k                    & Vue                   & No                & -                & \url{https://buefy.org/ }                                                        \\ \hline
    Vuesax                  & 5k                      & Vue                   & No                & -                & \url{https://vuesax.com/ }                                                       \\ \hline
    Chakra                  & 1.1                     & Vue                   & No                & -                & \url{https://vue.chakra-ui.com/ }                                                \\ \hline
    Fish UI                 & 867                     & Vue                   & No                & -                & \url{https://myliang.github.io/fish-ui/ }                                        \\ \hline
    Wired Elements          & 8.5k                    & Web Components        & Yes               & 1.0.0            & \url{https://wiredjs.com/}                                                       \\ \hline
    Clarity Design          & 6.2k                    & Web Components        & Yes               & 5.1.0            & \url{https://clarity.design/}                                                    \\ \hline
    Fast                    & 5.6k                    & Web Components        & Yes               & 1.8.0            & \url{https://www.fast.design/}                                                   \\ \hline
    Material Web Components & 2.5k                    & Web Components        & No                & -                & \url{https://github.com/material-components/material-components-web-components } \\ \hline
    UI5                     & 887                     & Web Components        & No                & -                & \url{https://sap.github.io/ui5-webcomponents/}                                   \\ \hline
    Vaadin                  & 17                      & Web Components        & No                & -                & \url{https://vaadin.com/}                                                        \\ \hline
    Onsen                   & 8.3k                    & Multi-Framework       & Yes               & 2.11.2           & \url{https://onsen.io/}                                                          \\ \hline
    Primefaces (Angular)    & 1.3k                    & Multi-Framework       & Yes               & 11.3.2-SNAPSHOT  & \url{https://www.primefaces.org/primeng/}                                        \\ \hline
    Primefaces (React)      & 1.3k                    & Multi-Framework       & Yes               & 6.2.2-SNAPSHOT   & \url{https://www.primefaces.org/primereact/}                                     \\ \hline
    Primefaces (Vue)        & 1.1k                    & Multi-Framework       & Yes               & 3.3.6-SNAPSHOT   & \url{https://www.primefaces.org/primevue/}                                       \\ \hline
    Syncfusion              & unknown (not on github) & Multi-Framework       & No (paid)         & 1.0.0            & \url{https://www.syncfusion.com/}
  \end{tabularx}
  \caption{Collected UI libraries, the number of github stars and whether they were included in the study}
  \label{tab:design:ui-libraries}
\end{table*}

\section{Analysis of results}
In order to compare the collected measurements, we use the median value over a set of measurements. Compared to the average, the median minimizes the impact of outliers in a data set. Since the measurements we collect are likely to have outliers in them due to the nature of time-sensitive measurements, this statistical value is likely to be a better choice.