As the number of JavaScript (JS) frameworks increases, a single cross-framework format is essential. With the introduction of Web Components, this format was finally available. However, fact remains that most JS component libraries currently are and continue to be written in some JS framework. As such, a method of converting these components from this JS framework to Web Components needs to be devised, such that the components can be used in any framework. In this paper, we will be presenting a case study where we convert a set of Angular components to Web Components. In evaluating the quality and performance of the resulting Web Components, we find them the page load time to be roughly twice as long, with render times of individual components being only being about 5ms slower for a single component, which is about twice as long. While this number appears high, the resulting render times remain competitive with the render times of various other component libraries. These render times increase to being roughly three times as long when the number of components is increased to 100. Additionally, we find the impact of the performed case study on both the codebase containing the source components and other developers to be minimal. These findings together lead us to the conclusion that the conversion of Angular components to Web Components is certainly feasible, with load and render times not increasing that much and with it being a time-saving method for businesses wanting to create a cross-framework compatible component library without re-writing a set of existing Angular components.