Since 2018, a set of technologies together referred to as Web Components are supported in all major browsers. Web Components are a set of technologies that enable support for the creation of custom HTML elements, thereby allowing for encapsulation of functionally and semantically related code. In this respect, they are similar to JavaScript (JS) frameworks, with the exception that these created elements (also called Web Components) work out of the box and do not require any additional code to function. Any component such as button, checkbox, or switch written using Web Components will work everywhere. This contrasts with code written using a JS framework, which will generally only work on web pages also using that JS framework. If a web page is using a different JS framework, the code will not work. 

There are various reasons for developers to convert existing an existing set of components (generally referred to as a component library) from a JS framework to Web Components. In this paper, we will be presenting a case study tackling such a scenario for one of these JS frameworks, namely the converting of a set of Angular components to Web Components. Angular is one of the more popular JS frameworks for building web applications. It also suffers from the previously mentioned issue of components written in Angular not being usable in other JS frameworks. The conversion to Web Components presents a solution to this issue.

In evaluating the quality and performance of the resulting Web Components, we find the load time of the JS code to be roughly twice as long, with render times of individual components being about 5ms slower for a single component, which is about twice as long. The resulting render times remain competitive with the render times of various other component libraries. Additionally, we find indications that the impact of the performed case study on the codebase containing the source components is minimal. These findings together lead us to conclude that the conversion of Angular components to Web Components is feasible. This conversion process presents a time-saving method for developers wishing to create a cross-framework component library based on an existing Angular component library.